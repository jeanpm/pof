\documentclass[12pt,a4paper,onecolumn]{amsart}
\usepackage[utf8]{inputenc}
\usepackage[english]{babel}
\usepackage{amsmath}
\usepackage{amsfonts}
\usepackage{amssymb}
\usepackage{fullpage}
\usepackage{tcolorbox}
\usepackage{microtype}
%\usepackage{helvet}
%\usepackage{bm}
%\usepackage{bbm}

\author{Jean P. Martins\\ \texttt{\lowercase{jeanp@dei.uc.pt}}}
\title{A conceptual model for optimization problems}
\date{\today}

\begin{document}

\begin{abstract}
Defines general concepts related to combinatorial optimization problems and illustrates how they can be used to implement metaheuristics in terms of elementar neighborhood-based operations.
\end{abstract}

\maketitle

\newtheorem{definition}{Definition}[section]
\newtheorem{example}{Example}[section]

\section{Definitions}

\begin{definition}[Component]
In a combinatorial optimization problem, a component $c\in C$ is the elementary unity of a solution. 
\end{definition}

\begin{definition}[Search space]
The search space $S$ of a combinatorial optimization problem is a subset of the power set of components $S_{\leq n}(2^C)$ (only $S$ from now on), in which elements $x\in S$ can be composed of a maximum of $n$ components. In practical terms, $S$ is implicitly defined by $C$ and $n$.
\end{definition}

\begin{tcolorbox}
	\begin{example}[Knapsack problem component]
	In a knapsack problem, a component is an item. There are $n$ available items (components) $C=\{1,2,\dots,n\}$. Therefore, elements $x\in S$ can be composed of a maximum of $n$ items, i.e. $|C| = n$. 
	\end{example}
	
	\begin{example}[Minimum spanning tree component]
	In a minimum spanning tree problem, a component is an edge. Considering a undirected complete graph $G=(V,E)$, with $|V|=n$, there are $|C| = {(n^2 - n)}/{2}$ available edges (components). 
	\end{example}
\end{tcolorbox}

\begin{definition}[Neighborhood]
A neighborhood function $N_{k}(x) = \{y\in S : d(x,y) = \delta(k)\}$ defines the neighbors of $x$ in terms of components, with $k\geq 0$ denoting the number of components in which $x$ and $y$ differ. The size of $N(x)$ is usually $|C|\in O(n^k)$.
\end{definition}

\begin{tcolorbox}
	\begin{example}[Add/remove neighborhood]
	Solutions $x,y\in S$ are neighbors if they differ in only one component, i.e. $k = 1$. A solution $y$ can be generated from $x$ by adding or removing a component.
	\end{example}
	
	\begin{example}[Swap neighborhood]
	Solutions $x,y\in S$ are neighbors if they differ in two components, i.e. $k = 2$. A solution $y$ can be generated from $x$ by removing a component and adding another component. 
	\end{example}
	
	\begin{example}[2-opt neighborhood]
	Solutions $x,y\in S$ are neighbors if they differ in two components, i.e. $k = 2$. A solution $y$ can be generated from $x$ by removing two components (edges) $(a,b)$ and $(c,d)$ and adding the components (edges) $(a,d)$ and $(c,b)$.
	\end{example}
\end{tcolorbox}

Observe that the neighborhoods add/remove and swap are independent of the component structure, whereas 2-opt assume the components are edges in a graph.

\section{Implementation details}

\begin{definition}[Search space]
The search space structure $S$ defines the list of components $C$ it supports.
\end{definition}


\begin{definition}[Solution]
Solutions $x$, as elements of the search space $S$, have a copy of the component list. The components composing $x$ are stored in $C_u = \{c\in C : c\in x\}$, the remaining (available) are stored in $C_a =\{ c\in C: c\notin x\}$. In summary, $C_u \cup C_a = C$.
\end{definition}

\begin{definition}[Neighborhood]
Neighboring solutions are generated by the function $N$, which has access to solutions $x$ and their component lists.
\end{definition}


\end{document}