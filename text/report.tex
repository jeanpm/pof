\documentclass[12pt,a4paper,onecolumn]{amsart}
\usepackage[utf8]{inputenc}
\usepackage[english]{babel}
\usepackage{amsmath}
\usepackage{amsfonts}
\usepackage{amssymb}
\usepackage{fullpage}
\usepackage{tcolorbox}

\author{Jean P. Martins}
\title{A conceptual model for optimization problems}

\begin{document}

\begin{abstract}

\end{abstract}

\maketitle

\newtheorem{definition}{Definition}[section]
\newtheorem{example}{Example}[section]

\begin{definition}[Component]
In a combinatorial optimization problem, a component $c\in C$ is the elementary unity of a solution. 
\end{definition}

\begin{definition}[Search space]
The search space $S$ of a combinatorial optimization problem is a subset of the powerset of components $S_{\leq n}(2^C)$ (only $S$ for simplicity), in which elements $x\in S$ can be composed of a maximum of $n$ components. In practical terms $S$ is implicitly defined by $C$ and $n$.
\end{definition}

\begin{tcolorbox}
	\begin{example}[Knapsack problem component]
	In a knapsack problem, a component is an item. There are $n$ available items (components) $C=\{1,2,\dots,n\}$. Therefore, elements $x\in S$ can be composed of a maximum of $n$ items. 
	\end{example}
	
	\begin{example}[Minimum spanning tree component]
	In a minimum spanning tree problem, a component is an edge. Considering a undirected complete graph $G=(V,E)$, with $|V|=n$, there are ${(n^2 +1)}/{2}$ available edges (components). 
	\end{example}
\end{tcolorbox}

\end{document}